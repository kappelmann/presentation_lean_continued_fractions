\documentclass{beamer}
\usepackage{appendixnumberbeamer}

\mode<presentation>{\usetheme[subsectionpage=progressbar,block=fill,numbering=none]{metropolis}}

\usepackage[english]{babel} 
\usepackage[utf8]{inputenc}

\usepackage{graphicx} % Allows including images
\usepackage{caption}
\usepackage{booktabs} % Allows the use of \toprule, \midrule and \bottomrule in tables
\usepackage{multicol} 

% Math packages
\usepackage{amsmath}
\usepackage{mathtools}
\usepackage{amssymb}
\usepackage{mathpartir}

% Lean 
\usepackage{color}
\definecolor{keywordcolor}{rgb}{0.7, 0.1, 0.1}   % red
\definecolor{commentcolor}{rgb}{0.4, 0.4, 0.4}   % grey
\definecolor{symbolcolor}{rgb}{0.0, 0.1, 0.6}    % blue
\definecolor{sortcolor}{rgb}{0.1, 0.5, 0.1}      % green
\usepackage{listings}
\def\lstlanguagefiles{lstlean.tex}
\lstset{language=lean}

% Coloured boxes
\usepackage{tcolorbox}
\colorlet{alert}{mLightBrown}
\newtcolorbox{alertbox}
{standard jigsaw, opacityback=0,colframe=alert}
\newtcolorbox{tbox}
{standard jigsaw, opacityback=0,opacityframe=0}

\title{Continued Fractions in Lean} % the title on the title page
\subtitle{A Newbie's Adventure}

\author{Kevin Kappelmann} % Your name
\institute[VU Amsterdam]{Vrije Universiteit Amsterdam}
\date{June 14, 2019} % Date, can be changed to a custom date

\begin{document}

\maketitle
%\begin{frame}{Overview}
	%\setbeamertemplate{section in toc}[sections numbered]
	%\setbeamertemplate{subsection in toc}{\leavevmode\leftskip=3.2em\rlap{\hskip-2em\inserttocsectionnumber.\inserttocsubsectionnumber}\inserttocsubsection\par}
	%\tableofcontents
%\end{frame}

%------------------------------------------------
\section{Let's Go on an Adventure}
\begin{frame}{Choose a Weapon}
\pause
\only<2>{\centering{\includegraphics[height=0.6\textheight]{img/choose_prover.png}}}
\only<3->{\centering{\includegraphics[height=0.6\textheight]{img/choose_prover2.png}}}
\visible<4>{
\centerline{\dots perhaps because I am interning at VU Amsterdam}
}
\end{frame}
%------------------------------------------------
\begin{frame}{The Adventurer's Skill Set}
\pause
\begin{itemize}[<+->]
\item Some experience using Isabelle
\item First project with a dependent type theorem prover
\item Basic maths and functional programming knowledge
\end{itemize}
\end{frame}
%------------------------------------------------
\begin{frame}
    \centering\includegraphics[height=0.8\textheight]{img/salt_shaker.png}
\end{frame}
%------------------------------------------------
\section{Bla}
\begin{frame}{Basic Definitions}
A generalized continued fraction is\dots
\pause
\begin{equation*}
b + \cfrac{a_0}{b_0 + \cfrac{a_1}{b_1 + \cfrac{a_2}{b_2 + \cfrac{a_3}{b_3 + \ddots\,}}}}
\end{equation*}
\pause
\begin{itemize}[<+->]
\item $b$ is called the \emph{integer part}
\item each $a_i$ is a \emph{partial numerator}
\item each $b_i$ is a \emph{partial denominator}
\end{itemize}
\end{frame}
%------------------------------------------------
\begin{frame}
\begin{equation*}
b + \cfrac{a_0}{b_0 + \cfrac{a_1}{b_1 + \cfrac{a_2}{b_2 + \cfrac{a_3}{b_3 + \ddots\,}}}}
\end{equation*}
\begin{lstlisting}
Some symbols: ℕ ℤ ∩ ⊂ ∀ ∃ Π α β γ ∈ ⦃ ⦄
\end{lstlisting}
\end{frame}
%------------------------------------------------
\section{End of the Story}
\begin{frame}{Choose a Weapon}
Summary. Things Learned.
\end{frame}
%------------------------------------------------
\begin{frame}[standout]
\center
\Large{Thanks for your attention! Any questions?}

\vspace{20mm}
\normalsize{Formalisation can be found at \url{github.com/kappelmann/lean-continued-fractions}}
\end{frame}
%----------------------------------------------------------------------------------------
%\begin{frame}[allowframebreaks]{References}
  %\bibliography{../paper/sources.bib}
  %\bibliographystyle{abbrv}
%\end{frame}

\begin{frame}[allowframebreaks]{Image Sources}
\begin{itemize}
\item Salt shaker: Modified from \url{bit.ly/2K8Jw8s}
\item Link 1: \url{bit.ly/2wMGOwE}
\item Link 2: \url{bit.ly/2RaypfX}
\end{itemize}
\end{frame}

\end{document} 
